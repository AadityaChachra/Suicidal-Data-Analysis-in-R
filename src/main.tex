\documentclass[12pt,a4paper]{article}
\usepackage[margin=1in]{geometry}
\usepackage{graphicx}
\usepackage{float}
\usepackage{booktabs}
\usepackage{hyperref}
\usepackage{caption}
\usepackage{subcaption}
\usepackage{amsmath}
\usepackage{amssymb}
\usepackage{setspace}
\usepackage{titlesec}
\usepackage{fancyhdr}

\hypersetup{
  colorlinks = true,
  urlcolor   = blue,
  linkcolor  = black,
  citecolor  = black
}

\titleformat{\section}[block]{\bfseries\Large\filcenter}{}{1em}{}
\titleformat{\subsection}[block]{\bfseries\large}{}{1em}{}

\pagestyle{fancy}
\fancyhf{}
\rhead{\thepage}
\lhead{Suicide Data Analysis Report}
\renewcommand{\headrulewidth}{0.4pt}

\onehalfspacing

\begin{document}

\begin{titlepage}
  \centering
  \vspace*{3cm}
  {\Huge \textbf{Suicide Data Analysis Report}}\\[2em]
  {\large \textbf{Computing Tools Workshop}}\\[0.5em]
  {\large \textbf{Mathematics and Computing}}\\[3em]
  {\large \textbf{Aman Kumar}}\\[0.3em]
  {\large \textbf{Roll No: 2022UCM2386}}\\
  \vfill
  \textbf{\today}
  \vspace*{2cm}
\end{titlepage}

\tableofcontents
\clearpage

\section{Introduction}
This report documents an end-to-end analysis of a suicide dataset. The main objectives include:
\begin{enumerate}
  \item Loading and cleaning the dataset (handling missing values and duplicates).
  \item Generating summary statistics grouped by country, year, gender, and age.
  \item Creating a new column for suicides per 100,000 population.
  \item Visualizing total suicides over time for the top five countries with the highest suicide rates.
  \item Generating heatmaps to display suicide rates by country and age group.
  \item Performing a correlation analysis between GDP per capita and suicide rates.
  \item Conducting a T-test to identify whether there's a significant difference in suicide rates between males and females.
  \item Presenting conclusions and key findings with visual evidence.
\end{enumerate}

\section{Data Explanation}
The primary dataset is stored at \texttt{Data/master.csv}. After extensive cleaning—removing duplicates, filtering out rows with missing suicides or population data, and converting data types—an intermediate cleaned dataset (\texttt{Data/clean\_suicide\_data.csv}) was produced. We then added a new column (\texttt{suicides\_per\_100k\_calc}) to represent the suicide count per 100,000 population, culminating in a final dataset saved as \texttt{clean\_suicide\_data\_with\_rate.csv} in the \texttt{output} directory.

\subsection{Cleaning Steps}
\begin{itemize}
  \item \textbf{Missing Values:} Rows with \texttt{NA} in critical columns (e.g., suicides, population) were removed.
  \item \textbf{Duplicates:} Identical rows were dropped to ensure uniqueness.
  \item \textbf{Type Conversion:} Columns like \texttt{year}, \texttt{suicides\_no}, and \texttt{population} were converted to numeric or integer formats.
  \item \textbf{GDP Fields:} \texttt{gdp\_for\_year} and \texttt{gdp\_per\_capita} were cleaned by removing special characters and converting them to numeric.
\end{itemize}

\section{Summary Statistics}
For each country, year, gender, and age group, we computed:
\begin{itemize}
  \item \textbf{Total Suicides:} Sum of \texttt{suicides\_no}.
  \item \textbf{Average Suicide Rate:} Mean of \texttt{suicides\_per\_100k\_calc}.
  \item \textbf{Average Population:} Mean of \texttt{population}.
\end{itemize}
These summaries allowed us to compare suicide tendencies across different demographics.

\section{Suicides per 100k Population}
A new column \texttt{suicides\_per\_100k\_calc} was created:
\[
\text{suicides\_per\_100k\_calc} = \frac{\text{suicides\_no}}{\text{population}} \times 100000
\]
This metric normalizes suicide counts relative to population sizes, enabling more equitable comparisons across countries and years.

\section{Visualization and Results}

\subsection{Top 5 Countries by Suicide Rate}
\begin{figure}[H]
  \centering
  \includegraphics[width=0.85\textwidth]{output/top5_countries_suicides_over_time.png}
  \caption{Total Suicides Over Time for the Top 5 Countries}
  \label{fig:top5}
\end{figure}

Figure~\ref{fig:top5} shows the total suicides by year for countries with the highest average suicide rates. Notably, some countries show consistently high totals, pointing to long-term societal or economic factors.

\subsection{Heatmaps by Age Group and Country}
To visualize how age groups differ among countries, we generated multiple heatmaps. Below are examples of different group subsets from the dataset:

\begin{figure}[H]
  \centering
  \begin{subfigure}{0.45\linewidth}
    \includegraphics[width=\linewidth]{output/heatmap_group_1.png}
    \caption{Heatmap Group 1}
  \end{subfigure}
  \hfill
  \begin{subfigure}{0.45\linewidth}
    \includegraphics[width=\linewidth]{output/heatmap_group_8.png}
    \caption{Heatmap Group 8}
  \end{subfigure}
  \vskip\baselineskip
  \begin{subfigure}{0.45\linewidth}
    \includegraphics[width=\linewidth]{output/heatmap_group_22.png}
    \caption{Heatmap Group 22}
  \end{subfigure}
  \hfill
  \begin{subfigure}{0.45\linewidth}
    \includegraphics[width=\linewidth]{output/heatmap_group_30.png}
    \caption{Heatmap Group 30}
  \end{subfigure}
  \caption{Age-Group-Based Heatmaps for Various Country Subsets}
  \label{fig:heatmaps}
\end{figure}

These heatmaps highlight variations in suicide rates across age brackets, indicating that certain age groups are at higher risk in specific countries.

\subsection{Correlation: GDP Per Capita vs. Suicide Rates}
\begin{figure}[H]
  \centering
  \includegraphics[width=0.65\textwidth]{output/gdp_vs_suicide_rate.png}
  \caption{Correlation Between GDP Per Capita and Suicide Rates}
  \label{fig:gdp_corr}
\end{figure}

Figure~\ref{fig:gdp_corr} shows a scatter plot with a fitted regression line. The correlation test indicated whether higher GDP per capita correlates positively or negatively with suicide rates. A significant p-value ($p < 0.05$) suggests a statistically meaningful relationship.

\subsection{T-Test Analysis: Gender Differences}
A two-sample T-test was performed:
\[
H_0: \mu_{\text{male}} = \mu_{\text{female}}, \quad
H_1: \mu_{\text{male}} \neq \mu_{\text{female}}
\]
\begin{figure}[H]
  \centering
  \includegraphics[width=0.65\textwidth]{output/TTest.png}
  \caption{Console Output of the T-Test Results}
  \label{fig:t_console}
\end{figure}

\begin{figure}[H]
  \centering
  \includegraphics[width=0.55\textwidth]{output/t_test_gender_boxplot.png}
  \caption{Boxplot of Suicide Rates by Gender}
  \label{fig:t_boxplot}
\end{figure}

If the p-value is below 0.05, there's a statistically significant difference in the mean suicide rate between males and females, as depicted in Figure~\ref{fig:t_boxplot}.

\subsection{Additional Distributions}
\begin{figure}[H]
  \centering
  \includegraphics[width=0.65\textwidth]{output/suicide_rate_distribution.png}
  \caption{Distribution of Suicide Rates (per 100k)}
  \label{fig:dist_suicides}
\end{figure}

Figure~\ref{fig:dist_suicides} shows the overall distribution of suicides per 100k. This histogram helps identify common ranges and extreme values.

\section{Interpretation and Conclusions}
\begin{itemize}
  \item \textbf{High-Risk Groups:} Heatmaps revealed certain age groups with elevated suicide rates, suggesting potential targeted prevention strategies.
  \item \textbf{GDP Correlation:} Economically prosperous nations may still suffer from significant mental health crises, though the direction and strength of correlation vary.
  \item \textbf{Gender Differences:} T-test results showed whether males or females are more at risk in certain contexts.
  \item \textbf{Policy Implications:} Findings can guide resource allocation for mental health programs across different demographic segments and economic statuses.
\end{itemize}

\section*{Author}
\textbf{Aman Kumar} \\
\textbf{Roll Number:} 2022UCM2386 \\
\textbf{Branch:} Mathematics and Computing \\
\textbf{Subject:} Computing Tools Workshop

\end{document}
